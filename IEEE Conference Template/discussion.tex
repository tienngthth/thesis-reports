This project successfully demonstrates the feasibility of integrating Prolog into GraalVM using the Projog library. Prolog facts are generated to represent multiple IR nodes in the compiler, and predicate rules are implemented to express the optimization logic. Canonicalization rules, such as those for AddNode, are simple, stateless, and use a singleton Projog class, resulting in consistent performance with minimal overhead. The first run is slower due to the initial class setup and rule loading, but later runs are faster. Conditional elimination is much slower because it must reinitialize the Projog engine and reload rules for every run. This is due to the need to manage dynamic facts and frequent creation and removal of Stamp objects, which represent known values of variables and are used to evaluate the correctness of nested conditions. This suggests that while Prolog integration may offer benefits in rule specification simplicity and clarity, it comes at a notable performance cost, especially for certain optimization phases. In addition, Projog itself is relatively slow, and performance could be improved by using a faster Prolog library.