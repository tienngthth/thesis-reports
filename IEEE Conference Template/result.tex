This project defines an operation throughput metric to evaluate the impact of integrating Prolog into GraalVM. An operation is defined as the process of building the IR graph for a method and applying all optimization phases, including canonicalization, loop invariant reassociation and conditional elimination. The operation is repeatedly performed for five seconds. The total number of completed operations is divided by five to calculate the average number of operations per second (operation throughput). This benchmark is conducted twice: once using the standard GraalVM (without Prolog), and once after incorporating Prolog into GraalVM. Comparing both results indicates whether Prolog improves or degrades the performance of the optimization process.

\begin{table}[h]
    \centering
    \fontsize{9pt}{9pt}
    \begin{tabular}{|l|l|l|}
        \hline
        $Test$ & $Without$ $Prolog$ & $Without$ $Prolog$ \\
        \hline
        $negateAdd$ & $2,242$ & $2,133$ \\
        $addNot$ & $2,964$ & $2,830$ \\
        $addNeutral$ & $3,005$ & $2,920$ \\
        $addSubNode$ & $3,050$ & $2,938$ \\
        \hline
    \end{tabular}
    \caption{Benchmark Results for Add Node Canonicalization Tests (Operations/Sec)}
    \label{table:Canonicalization}
\end{table} 
\smallbreak

\noindent  negateAdd:  $-y + x \rightarrow x - y$\\
addNot: $\thicksim x + x \rightarrow -1$\\
addNeutral: $0 + x \rightarrow x$\\
addSubNode: $(x - y) + y \rightarrow x$\\

\begin{table}[h]
    \centering
    \fontsize{9pt}{9pt}
    \begin{tabular}{|l|l|l|}
        \hline
        $Test$ & $Without$ $Prolog$ & $With$ $Prolog$ \\
        \hline
        $subSub$ & $304$ & $268$ \\
        $subAdd$ & $367$ & $330$ \\
        $addAdd$ & $375$ & $339$ \\
        $mulMul$ & $370$ & $334$ \\
        \hline
    \end{tabular}
    \caption{Benchmark Results for Reassociation Tests (Operations/Sec)}
    \label{table:Reassociation}
\end{table}
\smallbreak

\noindent subSub: $(i1 - ni) - i2 \rightarrow i1 - i2 - ni$\\
subAdd: $(i1 + ni) - i2 \rightarrow ni + i1 - i2$\\
addAdd: $(i1 + ni) + i2 \rightarrow ni + i1 + i2$\\
mulMul: $(i1 * ni) * i2 \rightarrow i1 * i2 * n$\\


The benchmark results for both the canonicalization and reassociation tests show only minor performance reductions when using Prolog. 
In the canonicalization tests (Table \ref{table:Canonicalization}), the performance drop is minimal, with only slight decreases in operations per second. 
For example, in the negateAdd test, throughput drops from 2,242 ops/sec to 2,133 ops/sec, and in the addNot test, it decreases from 2,964 ops/sec to 2,830 ops/sec. 
Similarly, in the reassociation tests (Table \ref{table:Reassociation}), the slowdown is relatively small, with a drop from 304 ops/sec to 268 ops/sec in the subSub test and from 375 ops/sec to 339 ops/sec in the addAdd test.
Overall, while there is some performance degradation, the impact of Prolog on both canonicalization and reassociation phases is not significant.

\begin{table}[h]
    \centering
    \fontsize{9pt}{9pt}
    \begin{tabular}{|l|l|l|}
        \hline
        $Test$ & $Without$ $Prolog$ & $With$ $Prolog$ \\
        \hline
        $condElim1$ & $1,661$ & $394$ \\
        $condElim2$ & $1,386$ & $355$ \\
        $condElim3$ & $951$ & $310$ \\
        $condElim4$ & $517$ & $243$ \\
        \hline
    \end{tabular}
    \caption{Benchmark Results for Condtional Eliminiation Tests (Operations/Sec)}
    \label{table:condElimination}
\end{table} 
\smallbreak

\noindent condElim1: 2 nested ifs.\\
condElim2: An outer if containing 2 child if statements, one of which contains a nested if.\\
condElim3: An outer if containing 3 child if statements, one of which contains a nested if.\\
condElim4: 8 nested ifs.\\

The results in Table \ref{table:condElimination} show a noticeable drop in optimization throughput when Prolog is used for conditional elimination. 
The number of operations per second is significantly lower with Prolog across all tests. 
For example, condElim4 drops from 517 to 243 ops/sec, and condElim1 from 1,661 to 394 ops/sec.
Moreover, the data indicates that performance degrades further as the nesting depth of if statements increases. 
This suggests that while the Prolog-based approach is functionally correct, it currently introduces performance costs. This is likely due to the complexity of interfacing with the Prolog engine and the overhead associated with managing deeply nested, stateful logic.