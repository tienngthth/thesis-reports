\chapter[Introduction]{Introduction}

Compiler optimizations improve program execution speed and overall performance, which can result in substantial cost savings for large-scale organizations, potentially reducing infrastructure expenses by millions annually. This paper examines the integration of GraalVM Ahead-of-Time (AOT) compiler optimizations with rule-based declarative logic programming languages. Logic programming languages allow programs to be analyzed efficiently through queries, enabling the implementation of code optimization that would otherwise be difficult to achieve.

Previous research has extensively utilized Datalog, a prominent logic programming language, for different levels of program analysis \cite{Bravenboer2009,Tonder2021,Lam2005,Benton2007}. However, there is a notable lack of evidence regarding the application of logic programming languages for the optimization and transformation of code. Diomidis’ work in 1999 explored an alternative method for expressing optimizations through declarative specifications, as opposed to traditional imperative code \cite{Spinellis1999}. However, Diomidis's work was limited to a rudimentary prototype of an optimizer with a "single optimization specification" and "limited flow-of-control optimizations" \cite{Spinellis1999}. Many DSL languages, often functional \cite{Hagedorn2020,Jonathan2018} were developed to describe optimization strategies in declarative ways. 

This research aims to assess the feasibility of implementing a Prolog-based optimizer within the GraalVM compiler framework. Prolog is among the earliest formal logic programming languages and continues to be one of the most widely used today. This study seeks to compare the expressiveness and efficacy of Prolog-based optimizations against the existing optimization techniques employed by GraalVM, and to investigate potential new optimizations enabled by Prolog specifications. The proposed approach involves expressing optimization rules using Prolog specifications and subsequently modifying the GraalVM compiler to query and utilize the Prolog database for optimization opportunities.
