\renewcommand{\arraystretch}{1.3} % Increase row height
\setlength{\tabcolsep}{10pt}      % Optional: increase column separation

\chapter[Performance Evaluation]{Performance Evaluation}
This project defines an operation throughput metric to evaluate the impact of integrating Prolog into GraalVM. An operation is defined as the process of building the IR graph for a method and applying all optimization phases, including canonicalization, loop invariant reassociation, and conditional elimination. The operation is repeatedly performed for five seconds. The total number of completed operations is divided by five to calculate the average number of operations per second (operation throughput). This gives a consistent and stable throughput metric measured in operations per second, where a higher value indicates faster execution time. Benchmark is conducted twice: once using the standard GraalVM (without Prolog), and once after incorporating Prolog into GraalVM.
\subsection*{Canonicalization Tests}

Test Descriptions:
\begin{itemize}
    \item \textit{negateAdd}: $-y + x \rightarrow x - y$
    \item \textit{addNot}: $\sim x + x \rightarrow -1$
    \item \textit{addNeutral}: $0 + x \rightarrow x$
    \item \textit{addSubNode}: $(x - y) + y \rightarrow x$
\end{itemize}

Test Results:
\begin{table}[h]
    \centering
    \begin{tabular}{|l|r|r|}
        \hline
        $Test$ & $Without$ $Prolog$ & $With$ $Prolog$ \\
        \hline
        $negateAdd$   & $2,242$ & $2,133$ \\
        $addNot$      & $2,964$ & $2,830$ \\
        $addNeutral$  & $3,005$ & $2,920$ \\
        $addSubNode$  & $3,050$ & $2,938$ \\
        \hline
    \end{tabular}
    \caption{Benchmark Results for Add Node Canonicalization Tests (Operations/Sec)}
    \label{table:Canonicalization}
\end{table}

\vspace{-10pt}

The benchmark results for the canonicalization tests (Table \ref{table:Canonicalization}) show a minor performance reductions when using Prolog. The performance drop is minimal, with only slight decreases in operations per second. For example, in the \textit{negateAdd} test, throughput drops from 2,242 ops/sec to 2,133 ops/sec, and in the \textit{addNot} test, it decreases from 2,964 ops/sec to 2,830 ops/sec.

\subsection*{Loop Invariant Reassociation Tests}
Test Descriptions:
\begin{itemize}
    \item \textit{subSub}: $(inv1 - i) - inv2 \rightarrow inv1 - inv2 - i$
    \item \textit{subAdd}: $(inv1 + i) - inv2 \rightarrow i + inv1 - inv2$
    \item \textit{addAdd}: $(inv1 + i) + inv2 \rightarrow i + inv1 + inv2$
    \item \textit{mulMul}: $(inv1 * i) * inv2 \rightarrow inv1 * inv2 * n$
\end{itemize}

Test Results:
\begin{table}[h]
    \centering
    \begin{tabular}{|l|r|r|}
        \hline
        $Test$ & $Without$ $Prolog$ & $With$ $Prolog$ \\
        \hline
        $subSub$ & $304$ & $268$ \\
        $subAdd$ & $367$ & $330$ \\
        $addAdd$ & $375$ & $339$ \\
        $mulMul$ & $370$ & $334$ \\
        \hline
    \end{tabular}
    \caption{Benchmark Results for Reassociation Tests (Operations/Sec)}
    \label{table:Reassociation}
\end{table}

\vspace{-10pt}

Similarly, the reassociation tests (Table \ref{table:Reassociation}) reveal a modest slowdown when Prolog is used. For instance, in the \textit{subSub} test, throughput drops from 304 ops/sec to 268 ops/sec, and in the \textit{addAdd} test, it decreases from 375 ops/sec to 339 ops/sec. Although there is some performance degradation, the effect of Prolog on the reassociation phase remains limited and does not severely impact throughput.

\subsection*{Conditional Elimination Tests}

Test Descriptions:
\begin{itemize}
    \item \textit{condElim1}: 2 nested ifs.
    \item \textit{condElim2}: An outer if containing 2 child if statements, one of which contains a nested if
    \item \textit{condElim3}: An outer if containing 3 child if statements, one of which contains a nested if.
    \item \textit{condElim4}: 8 nested ifs.
\end{itemize}

Test Results:
\begin{table}[h]
    \centering
    \begin{tabular}{|l|r|r|}
        \hline
        $Test$ & $Without$ $Prolog$ & $With$ $Prolog$ \\
        \hline
        $condElim1$ & $1,661$ & $394$ \\
        $condElim2$ & $1,386$ & $355$ \\
        $condElim3$ & $951$ & $310$ \\
        $condElim4$ & $517$ & $243$ \\
        \hline
    \end{tabular}
    \caption{Benchmark Results for Condtional Eliminiation Tests (Operations/Sec)}
    \label{table:condElimination}
\end{table} 
\smallbreak

\vspace{-10pt}
The results in Table \ref{table:condElimination} show a noticeable drop in optimization throughput when Prolog is used for conditional elimination. The number of operations per second is significantly lower with Prolog across all tests. For example, \textit{condElim1} drops from 1,661 to 394 ops/sec, and \textit{condElim4} from 517 to 243 ops/sec. \textit{condElim1} is the simplest and fastest test as it only has 2 nested ifs, which is reflected in the much higher throughput without Prolog. However, when applying Prolog, its throughput drops the most, and all test results become more uniform. This is largely explained by Prolog’s significant startup overhead, which imposes a fixed cost that dominates the runtime for smaller tests. As a result, this overhead “levels” the performance across different test complexities, causing a uniform but overall slower throughput.
