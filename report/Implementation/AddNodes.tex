\lstset{
    aboveskip=5pt,
    belowskip=5pt
}
\subsection{Add Node Canonicalization}
\subsection*{Use cases}
This project has implemented eight exploratory canonicalization rules for the \texttt{AddNode} as shown below.
\begin{align*}
    x + (-y)          &\rightarrow x - y        &\qquad -x + y           &\rightarrow y - x \\
    \sim x + x        &\rightarrow -1           &\qquad x + \sim x       &\rightarrow -1     \\
    x + 0             &\rightarrow x            &\qquad 0 + x            &\rightarrow x      \\
    (x - y) + y       &\rightarrow x            &\qquad x + (y - x)      &\rightarrow y      \\
\end{align*}
These rules target common arithmetic simplifications that occur in real-world programs and can lead to more efficient generated code by eliminating unnecessary operations.
Canonicalization is particularly valuable because it tends to expose further optimization opportunities and simplify control or data flow in the IR.
The Prolog rules for these transformations are simple and stateless, making them well-suited for logic-based, declarative expressions.
This declarative form makes the rules easy to read and reason about.

\subsection*{Prolog Implementation}
\begin{lstlisting}[language=Prolog]
% (x - y) + y -> x
canonical(node(add, node(Id, sub, IdX, IdY), IdY), Result) :-
   Result = lookup(IdX).
\end{lstlisting}
    
This rule matches an \texttt{addnode} where the first operand is a \texttt{subnode} identified by \texttt{Id}, and the second operand is identical to the right operand of that subtraction. The transformation relies solely on node identifiers rather than the internal structure of the nodes. When the same node ID \texttt{IdY} appears both as the right operand of the subtraction and as the second operand of the addition, the rule simplifies the expression \texttt{(x - y) + y} to just \texttt{x} by returning a lookup of \texttt{IdX}. Using node identifiers works correctly because the intermediate representation employs common subexpression sharing and global value numbering, ensuring that identical subexpressions share the same node ID. The resulting structure is returned as a new term in the variable \texttt{Result}, which is later parsed and reconstructed back into a GraalVM IR node by the result parser.

\subsection*{Java Implementation Comparision}
\begin{lstlisting}[language=Java]
if (forX instanceof NegateNode) {
    // -x + y => y - x
    return BinaryArithmeticNode.sub(forY, ((NegateNode) forX).getValue(), view);
} else if (forY instanceof NegateNode) {
    // x + -y => x - y
    return BinaryArithmeticNode.sub(forX, ((NegateNode) forY).getValue(), view);
} ...
\end{lstlisting}

The code snippet above provides example of the Java implementation for canonicalization rules.
Both the Java and Prolog versions aim to perform arithmetic simplification, and in this case, both are relatively simple because the underlying transformation rule itself is straightforward. 
However, the Prolog version stands out in terms of expressiveness and clarity. The Java code, while easy to follow, relies on verbose type checking and manual unwrapping of nodes, using imperative control flow and explicit casting to express the rule. In contrast, the Prolog version concisely encodes the same transformation using a single declarative clause that mirrors the algebraic identity directly. This leads to greater readability and a closer correspondence between the code and the mathematical reasoning behind the optimization, making the intent of the transformation immediately apparent.