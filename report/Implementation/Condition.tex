\subsection{Conditional Eliminationtion}
\subsection*{Use cases}
This project has implemented rules to match a variety of condtional elimination cases, examples of which are shown below.
\smallbreak

\begin{lstlisting}
// Case 1: X equals a constant
if (x == 1) {
    // this block is simplified to false
    if (x == 2) {}
}
// Case: 2: X is larger than a constant
if (x > 1) {
    // this block is simplified to false
    if (x == 0) {}
}
\end{lstlisting}

The Prolog rules for conditional elimination represent the most stateful and structurally complex analysis in this project. Unlike previous optimizations, which are generally stateless and operate locally on individual nodes, conditional elimination must track and manage global control flow context throughout the analysis. There are three related but distinct representations involved: the intermediate representation (IR) graph, the control flow graph (CFG), and the dominator tree. The IR graph models the program’s operations, while the CFG captures possible execution paths between blocks. The dominator tree is computed by Graal through a postorder traversal of the CFG. Although dominators can be computed on the fly, where each node’s dominator is either itself (if it is the start node) or the intersection of its parents’ dominators, this project uses Graal’s precomputed dominator tree and asserts blocks as facts into the Prolog knowledge base before querying. In this context, a block is represented as \texttt{block(BlockId, BeginNodeId, EndNodeId, DominatorBlockId)}, encoding a high-level basic block (\texttt{HIRBlock}) with its start and end nodes and a reference to its immediate dominator block in the control flow graph. For an if statement, the end node of the block is the actual if node that performs the branching. This approach simplifies queries by providing direct access to dominator relationships without recomputing them during analysis.

On the other hand, because neither the IR graph nor the dominator tree encode explicit parent and child links between outer and inner if nodes, the analysis must dynamically assert and retract facts describing current conditions and variable states as it traverses the dominator tree. A central concept in this analysis is the stamp. A stamp represents the known value range of a variable at a certain point in the control flow. A stamp is presented as \texttt{stamp(NodeId, UperBound, LowerBound, Level)}. Each stamp records the level of the if node where the value range was derived. This level indicates the depth of the condition in the nested control flow hierarchy, which is critical for defining the scope of the information. By knowing the level, the analysis can determine when a constraint no longer applies, such as when exiting a nested if block, and retract the stamp accordingly. Stamps are inferred from control flow conditions. For example, if the analysis encounters a branch that only executes when the variable x is less than five, it can assert a stamp recording that the value of x must be less than five on that path, and that this constraint originates at the current conditional node level. These known ranges are later used to simplify or eliminate subsequent conditional expressions, allowing the system to reason more effectively about the program’s behavior.

Because the analysis moves down through the dominator tree and later back up, the state must be carefully managed. When descending into a node, new stamps may be asserted to reflect updated knowledge about variable values. However, when the traversal moves upward again, those stamps must be retracted to avoid applying invalid assumptions to unrelated parts of the graph. This stateful approach ensures that the analysis remains sound and accurate, while still enabling powerful optimizations such as the elimination of unreachable branches and the simplification of guarded conditions.

\subsection*{Prolog Implementation}
\begin{lstlisting}[language=Prolog]
// Entry predicate for processing if nodes
process_if_nodes(Node, Result) :-
    node(NodeId, if, _, _, _),
    update_dominance_info(NodeId, DominatorId),
    check_successor_type(NodeId, DominatorId, SuccType),
    check_guard(NodeId, DominatorId, SuccType),
    try_fold(NodeId, DominatorId, SuccType, Result),
    Node = lookup(NodeId).
\end{lstlisting}

The predicate \texttt{process\_if\_nodes/2} identifies and processes all nodes in the knowledge base that represent conditional branches, i.e., nodes of type \texttt{if}. It does not take a specific input node but instead searches through the facts to find nodes matching the pattern \texttt{node(NodeId, if, \_, \_, \_)}.
For each such node, it updates the analysis state by determining the dominator node via \texttt{update\_dominance\_info/2}, which finds the controlling predecessor in the control flow graph. It then examines the type of successor nodes related to this dominator using \texttt{check\_successor\_type/3}, and inspects the guarding conditions with \texttt{check\_guard/3}. Based on these checks, \texttt{try\_fold/4} attempts to simplify or fold the conditional node.  When \texttt{process\_if\_nodes/2} is queried, it systematically returns pairs of \texttt{NodeId} and \texttt{Result}, such as \texttt{(NodeId = 1, Result = node(boolean, false))}, reflecting the outcome of attempting to optimize each conditional branch.

\smallbreak
\begin{lstlisting}[language=Prolog]
% Predicate to find dominator of a block
find_dominator(NodeId, DominatorId) :-
    block(_, _, NodeId, DominatorBlockId),
    block(DominatorBlockId, _, DominatorId, _).

% Update child node level
update_child_level(ParentNodeId, ChildNodeId) :-
    level(ParentNodeId, Level),
    NewLevel is Level + 1,
    asserta_unique(level(ChildNodeId, NewLevel)).

% Update global state
update_dominance_info(NodeId, DominatorId) :-
    find_dominator(NodeId, DominatorId) ->
    (
        % Update node level and retract stale stamp
        update_child_level(DominatorId, NodeId),
        retract_stale_stamps(NodeId)
    );
    % First if node doesn't have a dominator
    asserta_unique(level(NodeId, 0)),
    fail.
\end{lstlisting}

When visiting a new node, the \texttt{update\_dominance\_info/2} coordinates process to find the node dominator, its level in the tree and retract stale stamps. First, it attempts to find its dominator. 
The predicate \texttt{find\_dominator/2} finds the immediate dominator of a node by looking up the \texttt{block} information.
After finding the \texttt{block} containing the \texttt{NodeId}, the predicate retrieves its dominator block identifier \texttt{DominatorBlockId}.
It then locates the \texttt{block} with that ID and returns the if node's identifier corresponding to that dominator block as \texttt{DominatorId}. 
Once the dominator is found, \texttt{update\_child\_level/2} updates the child node’s level relative to its parent by fetching the parent’s current level, incrementing it by one, and recording this new level for the child node using \texttt{asserta\_unique/1}. 
If no dominator is found (such as for the root or first \texttt{if} node), it assigns level zero to the node and returns failure. 
This mechanism tracks the depth of nodes in the dominator tree or control flow hierarchy, which is critical for later analysis or optimizations.

\begin{lstlisting}[language=Prolog]
% Predicate to find the maximum levels of the tree
find_max_level(MaxLevel) :-
    findall(Level, level(_, Level), Levels),
    max_list(Levels, MaxLevel).

% Retract all stamps that are guarded by the lower node level
retract_stale_stamps(NodeId) :-
    level(NodeId, MinLevel),
    find_max_level(MaxLevel),
    findall(Num, between(MinLevel, MaxLevel, Num), StaleLevels),
    retract_stamps(StaleLevels).
retract_stamps([]).
retract_stamps([Level|Rest]) :-
    retractall(stamp(_, _, _, Level)),
    retract_stamps(Rest).
\end{lstlisting}


To ensure consistency in the analysis, stale or outdated information about nodes lower in the dominator tree must be removed when visiting a new node. The predicate \texttt{retract\_stale\_stamps/1} achieves this by first determining the level of the current node, then finding the maximum node level recorded so far. It collects all levels between the current node’s level and the maximum and retracts all corresponding \texttt{stamp/4} facts for those levels, effectively clearing cached data that may no longer be valid due to new information higher in the tree.

\newpage

\begin{lstlisting}[language=Prolog]
% Case: NodeId is the true successor
check_successor_type(NodeId, DominatorId, true) :-
    node(DominatorId, if, TrueSucc, _, _),
    node(TrueSucc, begin, NodeId).

% Case: NodeId is the false successor
check_successor_type(NodeId, DominatorId, false) :-
    node(DominatorId, if, _, FalseSucc, _),
    node(FalseSucc, begin, NodeId).

% Case: NodeId is neither (default case)
check_successor_type(_, _, unknown).
\end{lstlisting}

The above Prolog predicate \texttt{check\_successor\_type/3} determines whether a given \texttt{NodeId} is a true successor, a false successor, or neither (unknown) with respect to an \texttt{if} node identified by \texttt{DominatorId}. It starts by retrieving the \texttt{if} node corresponding to \texttt{DominatorId}, which has two successor nodes: \texttt{TrueSucc} and \texttt{FalseSucc}. Then it checks if the given \texttt{NodeId} matches the beginning node of the true branch (\texttt{TrueSucc}) in the first predicate; if so, it binds \texttt{Result} to \texttt{true}. Otherwise, it checks if \texttt{NodeId} matches the beginning node of the false branch (\texttt{FalseSucc}) in the second predicate; if so, it binds \texttt{Result} to \texttt{false}. If neither condition holds, it assigns \texttt{Result} to \texttt{unknown} in the final predicate.

\begin{lstlisting}[language=Prolog]
check_guard(DomOp, IdX, DomValueY, SuccType, NodeId) :-
    % Do nothing if dominator value is null
    DomValueY == null -> true;
    (
        level(NodeId, GuardId),
        member(DomOp,['==']), SuccType == true
            -> asserta_unique(stamp(IdX, DomValueY, DomValueY, GuardId));
        member(DomOp,['<']), SuccType == false
            -> asserta_unique(stamp(IdX, DomValueY, max_int, GuardId));
        true
    ).

% check case both conditions are binary conditions,
% have the same node x, and y of the dominator condition is a constant
check_guard(NodeId, DominatorId, SuccType) :-
    node(NodeId, if, _, _, ConditionId),
    node(ConditionId, _, IdX, _),
    node(IdX, parameter(_)),
    node(DominatorId, if, _, _, DominatorConditionId),
    node(DominatorConditionId, DominatorOp, IdX, DominatorY),
    node(DominatorY, constant(DominatorYValue, _)),
    check_guard(DominatorOp, IdX, DominatorYValue, SuccType, NodeId).
\end{lstlisting}

The above Prolog code is designed to recognize and remember useful information from conditional checks in earlier parts of a program’s control flow, which can later help optimize the program. The first predicate, \texttt{check\_guard/3}, operates at a higher level by examining the current node’s condition and its dominator’s condition. It ensures both conditions relate to the same variable with identifier \texttt{IdX} and that the dominator’s condition compares that variable to a constant. If these checks pass, it calls \texttt{check\_guard/5} to attempt recording a \texttt{stamp} representing the possible known variable values.
The second predicate, \texttt{check\_guard/5}, takes the comparison operator from the dominator node, the variable’s identifier \texttt{IdX}, the constant value \texttt{DomValueY} from the dominator node, the type of successor branch (true or false), and the current node \texttt{NodeId}. It first skips processing if the dominator value is missing. Otherwise, it retrieves the node’s level in the control flow and, based on the operator and successor type, records a \texttt{stamp} fact. For example, if the operator is equality and the branch is true, it records that the variable equals the constant. If the operator is less-than and the branch is false, it records that the variable is at least the constant value. This \texttt{stamp} acts as a fact that the system can later use to simplify expressions or optimize code.

\begin{lstlisting}[language=Prolog]
% Constant condition: value stamp available and deterministic
try_fold('>=', ValueY, MinX, MaxX, true) :- 
    ValueY \= null, MinX == MaxX, MinX >= ValueY.
try_fold('>=', ValueY, MinX, MaxX, false) :- 
    ValueY \= null, MinX == MaxX, MinX < ValueY.
...
try_fold(_, ValueY, _, _, unknown) :- ValueY \= null.
try_fold(_, ValueY, _, _, unknown) :- ValueY == null.
try_fold(_, _, _, _, unknown).  % General fallback

% try fold the condition
try_fold(NodeId, DominatorId, SuccType, Result) :-
    node(NodeId, if, _, _, ConditionId),
    node(DominatorId, if, _, _, DominatorConditionId),
    node(ConditionId, Op, IdX, IdY),
    node(DominatorConditionId, DominatorOp, IdX, DominatorY),
    node(IdX, parameter(_)),
    node(IdY, constant(ValueY, _)),
    node(DominatorY, constant(DominatorYValue, _)),
    (
        stamp(IdX, MinX, MaxX, _) ->
            try_fold(Op, ValueY, MinX, MaxX, Result);
    ).
\end{lstlisting}
\smallbreak
The above Prolog code attempts to simplify or “fold” conditional expressions by using known constant value ranges to decide the truth of conditions. The predicate \texttt{try\_fold/4} serves as the entry point: it extracts the operator and operands from the current conditional node and its dominator node, ensuring both involve the same variable \texttt{IdX} and that the compared values are constants. It then looks up any known value range for \texttt{IdX} stored as a \texttt{stamp}. If such a range exists, \texttt{try\_fold/4} calls the more specific \texttt{try\_fold/5} predicate to attempt to evaluate the condition conclusively as \texttt{true}, \texttt{false}, or \texttt{unknown}. \texttt{try\_fold/5}, defines specific cases for different operators such as \texttt{'>='}. For example, if the operator is \texttt{'>='}, the constant \texttt{ValueY} is not null, and the known value range for the variable is a single value (\texttt{MinX} equals \texttt{MaxX}), then it returns \texttt{true} if that value satisfies the condition or \texttt{false} otherwise. Other operators have similar specialized rules. If none of these match, the predicate defaults to \texttt{unknown}, indicating the condition cannot be conclusively evaluated.
A key limitation of the current code is that it only handles comparisons where a parameter is compared to a constant value; it does not support folding conditions involving two variables or more complex expressions.

\subsection*{Java Implementation Comparision}
Both the Prolog and Java implementations of conditional elimination are inherently complex due to the nature of the analysis, involving intricate state tracking and control flow reasoning. The Java version typically spans multiple classes and methods, spreading the logic across different parts of the codebase, which can make following the overall flow challenging without jumping between files. In contrast, the Prolog implementation, while lengthy, expresses the logic declaratively within a set of predicates, making the reasoning about pattern matching and rule application more direct. However, Prolog’s declarative style may require understanding predicate dependencies and backtracking, which can be less intuitive for those unfamiliar with logic programming. Overall, both versions balance complexity differently: Java through imperative, object-oriented constructs with explicit state management, and Prolog through concise logical rules with implicit control flow.