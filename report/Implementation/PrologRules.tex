\section{Prolog Optimization Rules}
\label{sec:rules}
Each optimization in this framework is defined in a Prolog file named after the transformation it performs, such as \texttt{addNode.pl}.
These files contain the logic rules that describe how specific intermediate representation (IR) patterns should be recognized and transformed during compilation.
Stateless optimizations such as canonicalization and loop invariant reassociation are purely pattern based.
They operate by matching structural patterns in the IR and returning transformed results without modifying the knowledge base.
In contrast, stateful optimizations such as conditional elimination require tracking intermediate state during query execution.
These optimizations use Prolog’s dynamic capabilities, including fact assertion and retraction, to simulate data flow and propagate information such as possible kwown variable values for conditional elimination optimization.

\lstset{
    aboveskip=5pt,
    belowskip=5pt
}
\subsection{Add Node Canonicalization}
\subsection*{Use cases}
This project has implemented eight exploratory canonicalization rules for the \texttt{AddNode} as shown below.
\begin{align*}
    x + (-y)          &\rightarrow x - y        &\qquad -x + y           &\rightarrow y - x \\
    \sim x + x        &\rightarrow -1           &\qquad x + \sim x       &\rightarrow -1     \\
    x + 0             &\rightarrow x            &\qquad 0 + x            &\rightarrow x      \\
    (x - y) + y       &\rightarrow x            &\qquad x + (y - x)      &\rightarrow y      \\
\end{align*}
These rules target common arithmetic simplifications that occur in real-world programs and can lead to more efficient generated code by eliminating unnecessary operations.
Canonicalization is particularly valuable because it tends to expose further optimization opportunities and simplify control or data flow in the IR.
The Prolog rules for these transformations are simple and stateless, making them well-suited for logic-based, declarative expressions.
This declarative form makes the rules easy to read and reason about.

\subsection*{Prolog Implementation}
\begin{lstlisting}[language=Prolog]
% (x - y) + y -> x
canonical(node(add, node(Id, sub, IdX, IdY), IdY), Result) :-
   Result = lookup(IdX).
\end{lstlisting}
    
This rule matches an \texttt{addnode} where the first operand is a \texttt{subnode} identified by \texttt{Id}, and the second operand is identical to the right operand of that subtraction. The transformation relies solely on node identifiers rather than the internal structure of the nodes. When the same node ID \texttt{IdY} appears both as the right operand of the subtraction and as the second operand of the addition, the rule simplifies the expression \texttt{(x - y) + y} to just \texttt{x} by returning a lookup of \texttt{IdX}. Using node identifiers works correctly because the intermediate representation employs common subexpression sharing and global value numbering, ensuring that identical subexpressions share the same node ID. The resulting structure is returned as a new term in the variable \texttt{Result}, which is later parsed and reconstructed back into a GraalVM IR node by the result parser.

\subsection*{Java Implementation Comparision}
\begin{lstlisting}[language=Java]
if (forX instanceof NegateNode) {
    // -x + y => y - x
    return BinaryArithmeticNode.sub(forY, ((NegateNode) forX).getValue(), view);
} else if (forY instanceof NegateNode) {
    // x + -y => x - y
    return BinaryArithmeticNode.sub(forX, ((NegateNode) forY).getValue(), view);
} ...
\end{lstlisting}

The code snippet above provides example of the Java implementation for canonicalization rules.
Both the Java and Prolog versions aim to perform arithmetic simplification, and in this case, both are relatively simple because the underlying transformation rule itself is straightforward. 
However, the Prolog version stands out in terms of expressiveness and clarity. The Java code, while easy to follow, relies on verbose type checking and manual unwrapping of nodes, using imperative control flow and explicit casting to express the rule. In contrast, the Prolog version concisely encodes the same transformation using a single declarative clause that mirrors the algebraic identity directly. This leads to greater readability and a closer correspondence between the code and the mathematical reasoning behind the optimization, making the intent of the transformation immediately apparent.
\lstset{
    aboveskip=0pt,
    belowskip=0pt
}
\subsection{And Node Canonicalization}
\subsection*{Use cases}
This project has implemented two canonicalization rules for the \texttt{AndNode} as shown below.
\begin{align*}
    x \,\&\, y &\rightarrow y \quad \text{if } \sim \texttt{mustSetX} \,\&\, \texttt{maySetY} = 0 \\
    x \,\&\, y &\rightarrow x \quad \text{if } \sim \texttt{mustSetY} \,\&\, \texttt{maySetX} = 0
\end{align*}

These rules, while still stateless and pattern-based, are slightly more involved than typical structural rules because they depend on semantic properties of the operands. Specifically, they require inspecting the \texttt{mustSet} and \texttt{maySet} bitmasks associated with the operands. In the context of compiler optimizations, \texttt{mustSet} and \texttt{maySet} are abstract representations of known and possible values at the bit level. The \texttt{mustSet} bitmask identifies bits that are guaranteed to be 1 in the operand, while the \texttt{maySet} bitmask identifies bits that could potentially be 1. These bitmasks enable reasoning about the outcome of bitwise operations without knowing exact operand values at compile time.
These rules help reduce redundant operations and enable further simplifications by eliminating unnecessary bitwise computations when the operand properties make them semantically redundant.

\subsection*{Prolog Implementation}
\begin{lstlisting}[language=Prolog]
% Complement of a bitmask
bitwise_not(X, Result) :-
    Mask is (1 << 32) - 1,
    Result is X xor Mask.

% x & y -> y if ~mustBeSetX & mayBeSetY == 0
canonical(node(andnode, X, Y, StampX, StampY), Result) :-
    StampX = stamp(_, _, MustBeSetX, _,),
    StampY = stamp(_, _, _, MayBeSetY),
    bitwise_not(MustBeSetX, ComplementMustBeSetX),
    (ComplementMustBeSetX /\ MayBeSetY) = 0,
    find_id(Y, IdY),
    Result = lookup(IdY).
\end{lstlisting}
This rule matches an \texttt{andnode} where the bitwise AND operation can potentially be simplified based on the bitmask metadata of its operands. 
It takes as input an \texttt{andnode} term consisting of two operand nodes along with their associated \texttt{stamp} information. 
Each \texttt{stamp} contains four pieces of data: minimum value that the node could have, maximum value that the node could have, \texttt{mustBeSet} bits, and \texttt{mayBeSet} bits. 
For this rule, only the \texttt{mustBeSet} and \texttt{mayBeSet} fields are relevant. 
The rule computes the bitwise complement of the first operand’s \texttt{mustBeSet} bits and then performs a bitwise AND with the second operand’s \texttt{mayBeSet} bits. 
If this result is zero, it indicates that the original AND operation can be simplified to just the second operand. 
The rule then extracts the unique identifier of this operand and returns it as the simplified result. 
The bitwise complement operation is implemented using a helper predicate \texttt{bitwise\_not} because the Projog engine does not provide a built-in predicate for this operation. 
This method leverages precise bit-level information to safely optimize the code.
A similar predicate is used to handle the complementary case where x is returned instead of y.
\subsection*{Java Implementation Comparision}
\begin{lstlisting}[language=Java]
IntegerStamp xStamp = (IntegerStamp) rawXStamp;
IntegerStamp yStamp = (IntegerStamp) rawYStamp;
if (((~xStamp.mustBeSet()) & yStamp.mayBeSet()) == 0) {
    return forY;
} else if (((~yStamp.mustBeSet()) & xStamp.mayBeSet()) == 0) {
    return forX;
}
\end{lstlisting}
The code snippet above provides the Java implementation for canonicalization rules.
Both the Java and Prolog versions implement the same underlying logic for the canonicalization of the bitwise AND operation, but they reflect their respective language paradigms in different ways. The Java code uses a single if-else statement to handle complementary cases within one procedural block, making the control flow explicit and linear. Meanwhile, the Prolog version expresses these cases as separate predicates, each capturing a specific rule declaratively. This separation aligns with Prolog’s logical programming style, allowing each case to be reasoned about independently. While the Prolog code is somewhat longer, it offers clarity through modular rules. Both approaches have similar control flow, but the expression differs to suit the strengths of each language.

\subsection{Loop Invariant Reassociation}
\subsection*{Use cases}
This project has implemented rules to match a variety of loop invariant reassociation patterns, as shown below.
\begin{align*}
    inv1 - (i + inv2)          &\rightarrow (inv1 - inv2) - i        &\qquad (i + inv2) - inv1           &\rightarrow i + (inv2 - inv1) \\
    (inv2 - i) + inv1          &\rightarrow (inv1 + inv2) - i        &\qquad (i - inv2) + inv1           &\rightarrow i + (inv1 - inv2) \\
    inv1 - (inv2 - i)          &\rightarrow i + (inv1 - inv2)        &\qquad inv1 - (i - inv2)           &\rightarrow (inv1 + inv2) - i \\
    (inv2 - i) - inv1          &\rightarrow (inv2 - inv1) - i        &\qquad (i - inv2) - inv1           &\rightarrow i - (inv1 + inv2) \\
    &\qquad  (i  \texttt{{\ }Op{\ }} inv2) \texttt{{\ }Op{\ }} inv1         &\rightarrow i \texttt{{\ }Op{\ }} (inv1 \texttt{{\ }Op{\ }} inv2) \\
\end{align*}
In these patterns, \texttt{inv1} and \texttt{inv2} represent invariant expressions that do not change within the loop, while \texttt{i} denotes a loop-variant variable. The goal of these rules is to isolate the loop-variant component in order to enable hoisting of loop-invariant computations outside the loop.
The last rule in the table represents a generalized reassociation pattern where both occurrences of the operator \texttt{Op} must be the same and drawn from a specific set of associative operations: addition, multiplication, bitwise AND, OR, XOR, and arithmetic max or min. 
These operations are associative and commutative, meaning the order of operands does not affect the result.

\subsection*{Prolog Implementation}
\begin{lstlisting}[language=Prolog]
% Entry point for invariant reassociation   
find_reassociate_inv(Node, LoopNodes, R) :-
    node_type(Node, NodeType, X, Y, IdX, IdY),
    find_reassociate(
        X, Y, IdX, IdY, 
        LoopNodes, Match1Id, Other1, MatchSide1
    ),
    node_type(
        Other1, Other1Type, Other1X, Other1Y, 
        Other1XId, Other1YId
    ),
    find_reassociate(
        Other1X, Other1Y, Other1XId, Other1YId, 
        LoopNodes, Match2Id, Other2, MatchSide2
    ),
    find_id(Other1, Other1Id),
    find_id(Other2, Other2Id),
    reassociate_rule(
        NodeType, Other1Type, Match1Id, Match2Id, 
        MatchSide1, MatchSide2, Other2Id, R
    ).
\end{lstlisting}

The predicate \texttt{find\_reassociate\_inv/3} serves as the entry point for identifying potential opportunities to reassociate invariant computations. The predicate takes a node (\texttt{Node}), a list of loop-related nodes (\texttt{LoopNodes}), and returns a reassociation result (\texttt{R}). It begins by extracting the type and subcomponents of the input node via \texttt{node\_type/6}. It then attempts to find a matching reassociation pattern for the first level of operands using \texttt{find\_reassociate/7}, which yields a matched node identifier (\texttt{Match1Id}), the non-matching "other" node (\texttt{Other1}), and which side the match occurred on (\texttt{MatchSide1}). The process is repeated for \texttt{Other1}, attempting to trace a second level of potential reassociation. The identifiers for these "other" nodes are resolved using \texttt{find\_id/2}, and finally, all gathered information is passed into \texttt{reassociate\_rule/8}, which checks whether a valid transformation rule applies and produces the reassociation result (\texttt{R}). Unlike the canonicalization rule, which typically operates on a single node in isolation, this rule explores a chain of connected nodes.
\smallbreak
\begin{lstlisting}[language=Prolog]
% Base case: NodeId should not be in LoopNodes.
is_invariant(Node, NodeId, LoopNodes) :-
    \+ member(NodeId, LoopNodes).

% Recursive case: Check invariants for the children nodes X and Y.
is_invariant(Node, NodeId, LoopNodes) :-
    node_type(Node, NodeType, X, Y, IdX, IdY),
    is_invariant(X, IdX, LoopNodes),
    is_invariant(Y, IdY, LoopNodes).

% Case: Left is invariant, Right is variant
find_reassociate(X, Y, IdX, IdY, LoopNodes, IdX, Y, left) :-
    is_invariant(X, IdX, LoopNodes),
    \+ is_invariant(Y, IdY, LoopNodes).

% Case: Right is invariant, Left is variant
find_reassociate(X, Y, IdX, IdY, LoopNodes, IdY, X, right) :-
    \+ is_invariant(X, IdX, LoopNodes),
    is_invariant(Y, IdY, LoopNodes).
\end{lstlisting}

\smallbreak
The above Prolog code defines logic for identifying loop-invariant computations and potential reassociation opportunities. The predicate \texttt{is\_invariant/3} determines whether a node is invariant with respect to a list of loop-related node identifiers given by \texttt{LoopNodes}. In the base case, a node is considered invariant if its identifier, \texttt{NodeId}, is not a member of \texttt{LoopNodes}. In the recursive case, the predicate assumes the node has two child nodes, \texttt{X} and \texttt{Y}, with corresponding identifiers \texttt{IdX} and \texttt{IdY}. It recursively checks that both children are invariant. The predicate \texttt{find\_reassociate/7} defines cases for recognizing expressions where one side is invariant and the other is not. In the first case, if the left child \texttt{X} is invariant and the right child \texttt{Y} is not, then it identifies \texttt{X} as the match and sets the side to \texttt{left}. In the second case, if the right child \texttt{Y} is invariant and the left child \texttt{X} is not, then it identifies \texttt{Y} as the match and sets the side to \texttt{right}. 
This logic enables detection of partial invariant expressions that could be reassociated to improve performance.

\smallbreak
\begin{lstlisting}[language=Prolog]
% Rule for addnode with subnode
reassociate_rule(
    addnode, subnode, Match1Id, Match2Id, 
    MatchSide1, MatchSide2, Other2Id, R
) :-
    reassociate_add_sub(Match1Id, Match2Id, Other2Id, MatchSide2, R).
\end{lstlisting}
\smallbreak
The above Prolog code defines a specific pattern-matching rule for reassociating arithmetic expressions in the case where the first operation is addition and the second operation is subtraction. The predicate \texttt{reassociate\_rule/8} captures this transformation by invoking \texttt{reassociate\_add\_sub/5}, which encodes two concrete transformation patterns.
The \texttt{reassociate\_rule/8} predicate takes as input the types of the first and second operations (e.g., \texttt{addnode} and \texttt{subnode}), the identifiers of two matched subexpressions (\texttt{Match1Id} and \texttt{Match2Id}), information about which sides of the expressions matched (\texttt{MatchSide1} and \texttt{MatchSide2}), the identifier of the other node involved in the reassociation (\texttt{Other2Id}), and finally returns the reassociated expression \texttt{R}. This input allows \texttt{reassociate\_rule/8} to select and apply the appropriate transformation pattern based on the structure and position of the invariant subexpressions.

\smallbreak
\begin{lstlisting}[language=Prolog]
% (inv2 - i) + inv1 -> (inv1 + inv2) - i
reassociate_add_sub(
    Match1Id, Match2Id, Other2Id, left,
    node(
        subnode, 
        node(addnode, lookup(Match1Id), lookup(Match2Id)), 
        lookup(Other2Id))
    ).

% (i - inv2) + inv1 -> i + (inv1 - inv2)
reassociate_add_sub(
    Match1Id, Match2Id, Other2Id, right,
    node(
        addnode, 
        node(subnode, lookup(Match1Id), lookup(Match2Id)), 
        lookup(Other2Id))
    ).
\end{lstlisting}

\smallbreak
The two clauses of \texttt{reassociate\_add\_sub/5} define transformation rules that restructure arithmetic expressions to group loop-invariant computations. The first clause handles the case where the second invariant appears on the left side of a subtraction, while the second clause deals with the scenario where the invariant is on the right. In both cases, the transformation rearranges the expression so that the invariant parts are grouped together. 
This is just one example of a family of rules used to optimize expressions through reassociation. Similar predicates exist for other operator combinations such as \texttt{reassociate\_sub\_sub}, \texttt{reassociate\_sub\_add}, and so on, each encoding the specific transformation rules for their respective operator pairings.

\subsection*{Java Implementation Comparision}

\begin{lstlisting}[language=Java]
ValueNode m1 = match1.getValue(node);
ValueNode m2 = match2.getValue(other);
ValueNode a = match2.getOtherValue(other);
if (isNonExactAddOrSub(node)) {
    ValueNode associated;
    if (invertM1) {
        associated = BinaryArithmeticNode.sub(m2, m1, view);
    } else if (invertM2) {
        associated = BinaryArithmeticNode.sub(m1, m2, view);
    } else {
        associated = BinaryArithmeticNode.add(m1, m2, view);
    }
    if (invertA) {
        return BinaryArithmeticNode.sub(associated, a, view);
    }
    if (aSub) {
        return BinaryArithmeticNode.sub(a, associated, view);
    }
    return BinaryArithmeticNode.add(a, associated, view);
}
...
\end{lstlisting}

The code snippet above provides the Java implementation for reassociating invariant in loop.
Both the Java and Prolog implementations perform loop invariant reassociation but differ significantly in style and clarity. The Java version uses several intermediary boolean variables like \texttt{invertM1}, \texttt{invertM2}, and \texttt{aSub} to control the flow and decide which arithmetic operation to apply. This approach, while compact, can be harder to follow because the logic is spread across multiple conditions and temporary variables, making the reasoning less direct. In contrast, the Prolog version is generally longer since it encodes each specific reassociation case as a separate predicate clause. This explicit case-by-case structure makes the Prolog code easier to understand and reason about, as each rule clearly corresponds to a particular algebraic transformation without relying on mutable state or complex branching.
\subsection{Conditional Eliminationtion}
\subsection*{Use cases}
This project has implemented rules to match a variety of condtional elimination cases, examples of which are shown below.
\smallbreak

\begin{lstlisting}
// Case 1: X equals a constant
if (x == 1) {
    // this block is simplified to false
    if (x == 2) {}
}
// Case: 2: X is larger than a constant
if (x > 1) {
    // this block is simplified to false
    if (x == 0) {}
}
\end{lstlisting}

The Prolog rules for conditional elimination represent the most stateful and structurally complex analysis in this project. Unlike previous optimizations, which are generally stateless and operate locally on individual nodes, conditional elimination must track and manage global control flow context throughout the analysis. The underlying intermediate representation graph is structured according to a dominator tree, which describes which nodes dominate others in the control flow. However, this tree does not provide a direct parent and child relationship that is easily accessible. Therefore, the analysis cannot rely solely on static relationships; it must dynamically assert and retract facts that describe current conditions and known variable states as it traverses the dominator tree.

A central concept in this analysis is the stamp. A stamp represents the known value range of a variable at a certain point in the control flow, typically recorded as a lower and upper bound. In addition to the bounds, each stamp also records the identifier of the node where the value range was derived. This allows the analysis to later determine the scope of that information and retract it when it becomes invalid. Stamps are inferred from control flow conditions. For example, if the analysis encounters a branch that only executes when the variable x is less than five, it can assert a stamp recording that the value of x must be less than five on that path, and that this constraint originates from the current conditional node. These known ranges are later used to simplify or eliminate subsequent conditional expressions, allowing the system to reason more effectively about the program’s behavior.

Because the analysis moves down through the dominator tree and later back up, the state must be carefully managed. When descending into a node, new stamps may be asserted to reflect updated knowledge about variable values. However, when the traversal moves upward again, those stamps must be retracted to avoid applying invalid assumptions to unrelated parts of the graph. This stateful approach ensures that the analysis remains sound and accurate, while still enabling powerful optimizations such as the elimination of unreachable branches and the simplification of guarded conditions.

\subsection*{Prolog Implementation}
\begin{lstlisting}[language=Prolog]
// Entry predicate for processing if nodes
process_if_nodes(Node, Result) :-
    node(NodeId, if, _, _, _),
    update_state(NodeId, DominatorId),
    check_successor_type(NodeId, DominatorId, SuccType),
    check_guard(NodeId, DominatorId, SuccType),
    try_fold(NodeId, DominatorId, SuccType, Result),
    Node = lookup(NodeId).
\end{lstlisting}

The predicate \texttt{process\_if\_nodes/2} identifies and processes all nodes in the knowledge base that represent conditional branches, i.e., nodes of type \texttt{if}. It does not take a specific input node but instead searches through the facts to find nodes matching the pattern \texttt{node(NodeId, if, \_, \_, \_)}.
For each such node, it updates the analysis state by determining the dominator node via \texttt{update\_state/2}, which finds the controlling predecessor in the control flow graph. It then examines the type of successor nodes related to this dominator using \texttt{check\_successor\_type/3}, and inspects the guarding conditions with \texttt{check\_guard/3}. Based on these checks, \texttt{try\_fold/4} attempts to simplify or fold the conditional node.  When \texttt{process\_if\_nodes/2} is queried, it systematically returns pairs of \texttt{NodeId} and \texttt{Result}, such as \texttt{(NodeId = 1, Result=node(boolean, false))}, reflecting the outcome of attempting to optimize each conditional branch.

\smallbreak
\begin{lstlisting}[language=Prolog]
% Predicate to find dominator of a block
find_dominator(NodeId, DominatorId) :-
    block(_, _, NodeId, DominatorBlockId),
    block(DominatorBlockId, _, DominatorId, _).

% Update child node level
update_child_level(ParentNodeId, ChildNodeId) :-
    level(ParentNodeId, Level),
    NewLevel is Level + 1,
    asserta_unique(level(ChildNodeId, NewLevel)).

% Update global state
update_state(NodeId, DominatorId) :-
    find_dominator(NodeId, DominatorId) ->
    (
        % Update node level and retract stale stamp
        update_child_level(DominatorId, NodeId),
        retract_stale_stamps(NodeId)
    );
    % First if node doesn't have a dominator
    asserta_unique(level(NodeId, 0)),
    fail.
\end{lstlisting}
\smallbreak
When visiting a new node, the \texttt{update\_state/2} coordinates process to find the node dominator, its level in the tree and retract stale stamps. First, it attempts to find its dominator. 
The predicate \texttt{find\_dominator/2} finds the immediate dominator of a node by looking up the block information: it first finds the block containing the \texttt{NodeId} and retrieves its dominator block, then finds the node corresponding to that dominator block, returning its identifier as \texttt{DominatorId}. 
Once the dominator is found, \texttt{update\_child\_level/2} updates the child node’s level relative to its parent by fetching the parent’s current level, incrementing it by one, and recording this new level for the child node using \texttt{asserta\_unique/1}. 
If no dominator is found (such as for the root or first \texttt{if} node), it assigns level zero to the node and returns failure. 
This mechanism tracks the depth of nodes in the dominator tree or control flow hierarchy, which is critical for later analysis or optimizations.
\smallbreak
\begin{lstlisting}[language=Prolog]
% Predicate to find the maximum value of levels
find_max_level(MaxLevel) :-
    findall(Level, level(_, Level), Levels),
    max_list(Levels, MaxLevel).

% Retract all stamps that are guarded by the lower node level
retract_stale_stamps(NodeId) :-
    level(NodeId, MinLevel),
    find_max_level(MaxLevel),
    findall(Num, between(MinLevel, MaxLevel, Num), StaleLevels),
    retract_stamps(StaleLevels).
retract_stamps([]).
retract_stamps([Level|Rest]) :-
    retractall(stamp(_, _, _, Level)),
    retract_stamps(Rest).
\end{lstlisting}

\smallbreak
To ensure consistency in the analysis, stale or outdated information about nodes lower in the dominator tree must be removed when visiting a new node. The predicate \texttt{retract\_stale\_stamps/1} achieves this by first determining the level of the current node, then finding the maximum node level recorded so far. It collects all levels between the current node’s level and the maximum and retracts all corresponding \texttt{stamp/4} facts for those levels, effectively clearing cached data that may no longer be valid due to new information higher in the tree.

\smallbreak

\begin{lstlisting}[language=Prolog]
% Predicate to check if a node is a true successor or a false successor of a given node
check_successor_type(NodeId, DominatorId, Result) :-
    node(DominatorId, if, TrueSucc, FalseSucc, _),
    node(TrueSucc, begin, NodeId) -> Result = true;
    node(FalseSucc, begin, NodeId) -> Result = false;
    Result = unknown.
\end{lstlisting}

\newpage
The above Prolog predicate \texttt{check\_successor\_type/3} determines whether a given \texttt{NodeId} is a true successor, a false successor, or neither (unknown) with respect to an \texttt{if} node identified by \texttt{DominatorId}. It starts by retrieving the \texttt{if} node corresponding to \texttt{DominatorId}, which has two successor nodes: \texttt{TrueSucc} and \texttt{FalseSucc}. Then it checks if the given \texttt{NodeId} matches the beginning node of the true branch (\texttt{TrueSucc}); if so, it binds \texttt{Result} to \texttt{true}. Otherwise, it checks if \texttt{NodeId} matches the beginning node of the false branch (\texttt{FalseSucc}); if so, it binds \texttt{Result} to \texttt{false}. If neither condition holds, it assigns \texttt{Result} to \texttt{unknown}.

\smallbreak
\begin{lstlisting}[language=Prolog]
check_guard(DomOp, IdX, DomValueY, SuccType, NodeId) :-
    % Do nothing if dominator value is null
    DomValueY == null -> true;
    (
        level(NodeId, GuardId),
        member(DomOp,['==']), SuccType == true
            -> asserta_unique(stamp(IdX, DomValueY, DomValueY, GuardId));
        member(DomOp,['<']), SuccType == false
            -> asserta_unique(stamp(IdX, DomValueY, max_int, GuardId));
        true
    ).

% check case both conditions are binary conditions,
% have the same node x, and y of the dominator condition is a constant
check_guard(NodeId, DominatorId, SuccType) :-
    node(NodeId, if, _, _, ConditionId),
    node(ConditionId, _, IdX, _),
    node(IdX, parameter(_)),
    node(DominatorId, if, _, _, DominatorConditionId),
    node(DominatorConditionId, DominatorOp, IdX, DominatorY),
    node(DominatorY, constant(DominatorYValue, _)),
    check_guard(DominatorOp, IdX, DominatorYValue, SuccType, NodeId).
\end{lstlisting}
\smallbreak
The above Prolog code is designed to recognize and remember useful information from conditional checks in earlier parts of a program’s control flow, which can later help optimize the program. The first predicate, \texttt{check\_guard/3}, operates at a higher level by examining the current node’s condition and its dominator’s condition. It ensures both conditions relate to the same variable with identifier \texttt{IdX} and that the dominator’s condition compares that variable to a constant. If these checks pass, it calls \texttt{check\_guard/5} to attempt recording a \texttt{stamp} representing the possible known variable values.
The second predicate, \texttt{check\_guard/5}, takes the comparison operator from the dominator node, the variable’s identifier \texttt{IdX}, the constant value \texttt{DomValueY} from the dominator node, the type of successor branch (true or false), and the current node \texttt{NodeId}. It first skips processing if the dominator value is missing. Otherwise, it retrieves the node’s level in the control flow and, based on the operator and successor type, records a \texttt{stamp} fact. For example, if the operator is equality and the branch is true, it records that the variable equals the constant. If the operator is less-than and the branch is false, it records that the variable is at least the constant value. This \texttt{stamp} acts as a fact that the system can later use to simplify expressions or optimize code.

\smallbreak
\begin{lstlisting}[language=Prolog]
% Constant condition: value stamp available and deterministic
try_fold('>=', ValueY, MinX, MaxX, true) :- 
    ValueY \= null, MinX == MaxX, MinX >= ValueY.
try_fold('>=', ValueY, MinX, MaxX, false) :- 
    ValueY \= null, MinX == MaxX, MinX < ValueY.
...
try_fold(_, ValueY, _, _, unknown) :- ValueY \= null.
try_fold(_, ValueY, _, _, unknown) :- ValueY == null.
try_fold(_, _, _, _, unknown).  % General fallback

% try fold the condition
try_fold(NodeId, DominatorId, SuccType, Result) :-
    node(NodeId, if, _, _, ConditionId),
    node(DominatorId, if, _, _, DominatorConditionId),
    node(ConditionId, Op, IdX, IdY),
    node(DominatorConditionId, DominatorOp, IdX, DominatorY),
    node(IdX, parameter(_)),
    node(IdY, constant(ValueY, _)),
    node(DominatorY, constant(DominatorYValue, _)),
    (
        stamp(IdX, MinX, MaxX, _) ->
            try_fold(Op, ValueY, MinX, MaxX, Result);
    ).
\end{lstlisting}
\smallbreak
The above Prolog code attempts to simplify or “fold” conditional expressions by using known constant value ranges to decide the truth of conditions. The predicate \texttt{try\_fold/4} serves as the entry point: it extracts the operator and operands from the current conditional node and its dominator node, ensuring both involve the same variable \texttt{IdX} and that the compared values are constants. It then looks up any known value range for \texttt{IdX} stored as a \texttt{stamp}. If such a range exists, \texttt{try\_fold/4} calls the more specific \texttt{try\_fold/5} predicate to attempt to evaluate the condition conclusively as \texttt{true}, \texttt{false}, or \texttt{unknown}. \texttt{try\_fold/5}, defines specific cases for different operators such as \texttt{'>='}. For example, if the operator is \texttt{'>='}, the constant \texttt{ValueY} is not null, and the known value range for the variable is a single value (\texttt{MinX} equals \texttt{MaxX}), then it returns \texttt{true} if that value satisfies the condition or \texttt{false} otherwise. Other operators have similar specialized rules. If none of these match, the predicate defaults to \texttt{unknown}, indicating the condition cannot be conclusively evaluated.

\subsection*{Java Implementation Comparision}
Both the Prolog and Java implementations of conditional elimination are inherently complex due to the nature of the analysis, involving intricate state tracking and control flow reasoning. The Java version typically spans multiple classes and methods, spreading the logic across different parts of the codebase, which can make following the overall flow challenging without jumping between files. In contrast, the Prolog implementation, while lengthy, expresses the logic declaratively within a set of predicates, making the reasoning about pattern matching and rule application more direct. However, Prolog’s declarative style may require understanding predicate dependencies and backtracking, which can be less intuitive for those unfamiliar with logic programming. Overall, both versions balance complexity differently: Java through imperative, object-oriented constructs with explicit state management, and Prolog through concise logical rules with implicit control flow.